\Chapter{Bevezetés}

% TODO: A bevezetésnek egy 1-2 oldalas áttekintésnek kellene lennie (alpontok nélkül), amelyben inkább a motiváció és a témakör aktualitása szerepel csak.

\Section{Integrált Fejlesztői Környezetek (IDE-k)}

Az integrált fejlesztői környezet, vagy röviden \emph{IDE} (az angol \emph{Integrated Development Environment} megnevezésből) 
egy olyan szoftver, amely elősegíti a felhasználót (ez esetben elsősorban programozót) más szoftverek fejlesztésében. 
Egy minimális IDE tartalmaz legalább egy forrásfájl-szerkesztőt, automatikus fordítási lehetőségeket, 
és hibakereső eszközöket (\textit{debugger}-t). A mai IDE-k nagyszámú egyéb szolgáltatásokat is nyújtanak a felhasználónak.
Például intelligens kódkiegészítést biztosítanak már az új kifejezés gépelésének megkezdésekor, támogatást adnak külső verziókövető szoftverekhez, illetve fordítást nem igénylő kódellenőrzést is végeznek a projektekben.

Nyilvánvaló, hogy miért előnyös egy IDE létezése egy megadott nyelvhez. Az előzőekben felsorolt szolgáltatásokat fejlesztő környezet nélkül a programozó vagy csak különféle, egymáshoz nem feltétlenül integrált szoftverekkel érhetné el 
(például nyelvspecifikus fordítóval, általános szövegszerkesztővel), 
vagy nem érhetné el egyáltalán (mint például az intelligens kódkiegészítést).

A szövegszerkesztőknél meg kell említenünk, hogy ma már a legtöbbjük intelligens eléggé ahhoz, 
hogy egy IDE funkcióinak egy részét maguk is szolgáltatni tudják.
Ilyen szövegszerkesztőket, és a támogatott funkcióikat a következő részben, \myaref{sec:ides} alfejezetben fogjuk megtekinteni. A dolgozat rámutat arra, hogy ezek a megoldások miért nem elégségesek.

\Section{Elérhető Rust fejlesztőkörnyezetek}
\label{sec:ides}

Mint azt már említettük, sok szövegszerkesztő ajánl IDE-szerű szolgáltatásokat Rust-hoz. 
A Rust közössége ezeket le is jegyzi, és egy nyilvánosan elérhető, 
\emph{Are we (I)DE yet?} című weboldalon összegezi \cite{ideyet}. A weboldalon láthatjuk a Rust-hoz elérhető legelterjedtebb szövegszerkesztőket, úgy mint például az \textit{Atom}, a \textit{Sublime Text} és a \textit{Visual Studio Code} nevű szerkesztőket. 
A legutóbbi kiemelendő, mivel a funkciókat bemutató táblázat alapján látszólag minden meg van benne, ami alapelvárásként szerepelt egy fejlesztőkörnyezettel szemben.

A szövegszerkesztők egyszerre több nyelvhez tudnak ilyen szolgáltatásokat nyújtani. 
Bár néhány szolgáltatás a szövegszerkesztőtől magától függ csak, a legtöbbhöz segítséget nyújt egy \emph{nyelvszerver} (\emph{language server}). 
A nyelvszerver egy olyan szoftver, ami a \emph{nyelvszerver protokoll} alapján információt nyújt IDE-knek, szövegszerkesztőknek, és egyéb programoknak a hozzá tartozó nyelvről \cite{lsp}.

Mivel a szövegszerkesztők bármilyen nyelvhez támogatást nyújthatnak egy közös protokoll által, így nyilvánvaló, hogy bármilyen szolgáltatás, ami a protokollon kívül esik, nem lesz elérhető egy nyelvhez sem. 
Van továbbá néhány funkció, amit viszont a szövegszerkesztőtől magától nem várhatunk el. Ilyen például egy grafikus szerkesztő, amivel szerkeszthetjük az alkalmazásunk grafikus felhasználói felületének elemeit, messze áll a szövegszerkesztéstől, és nehezen értelmezhető programozási nyelvtől van környezettől függetlenül egy szövegszerkesztőben. 
Ne feledjük, hogy a nyelvtámogatás maga csak egy mellékfunkció.
A szövegszerkesztő attól függetlenül szöveget még mindig tud szerkeszteni.

Egy másik megoldás, amit észrevehetünk az említett weboldalon, az az, hogy a Rust támogatást meglévő IDE-kbe építjük bele, mint például az \textit{Eclipse}, \textit{IntelliJ IDEA}, vagy \textit{Visual Studio} fejlesztőkörnyezetekbe. 
Bár azt gondolnánk, hogy ez közelebb áll a célunkhoz, valójában néhány szempontból pontosan annak az ellentéte igaz. Ilyen esetben a Rust, mint másodrendű nyelv kerül bele az alkalmazásba.
\textit{Eclipse} és \textit{IntelliJ IDEA} esetén a Java alá, \textit{Visual Studio} esetén a C\# alá főként. 
Észrevehetjük, hogy támogatás szempontjából nem a nyelvi támogatás lett komplexebb, hanem a szoftver maga.
A Rust szempontjából egy 'felturbózott' szövegszerkesztővel dolgozunk, és nem egy IDE-vel.

Legutoljára láthatjuk, hogy a Rust-hoz készült már néhány IDE. Ezek például a \emph{Ride} \cite{ride} és a \emph{SolidOak} \cite{solidoak}. 
Sajnálatos módon ezek sem felelnek meg a követelményeinknek. Nem csak, hogy mindkettő gyérebb a funkcionalitás oldalon, mint egy mezei szövegszerkesztő, de továbbá mindkét projekt halott is. 
A szakdolgozat írása idejében a \textit{Ride} legutolsó módosítása a projekt megszüntetésének bejelentése, és nem ajánlott produkciós környezetben, a \textit{SolidOak} Github oldala pedig archiválva lett.

\Section{Egy Rust IDE fejlesztése}

Az előző elemzésekből kitűnhet, hogy nyelvi szinten a Rust készen áll már akár most is egy Rust-centrikus fejlesztőkörnyezetre. 
%Továbbá, a Rust Forge oldalán látható, hogy tervekben áll\footnote{\texttt{https://forge.rust-lang.org/ides.html}} egy hivatalos daemon fejlesztése, ami tovább segíteni a leendő IDE-k munkáját egy Rust projekt kezelésében, és a lefordító módosítása, hogy támogassa az inkrementális és lusta fordítást, és hogy különféle hibák ellenére is végigmenjen a fordítási folyamaton.
A feladat így elsősorban egy olyan fejlesztőkörnyezet létrehozása, amely a Rust-ot elsőrendű nyelvként kezeli, figyelembe veszi annak speciális vonásait.
Fontos, hogy az IDE kényelmessé tegye a nyelvben való programozást, így a fejlesztés során gyakori kell, hogy legyen az iterálás a felhasználó munkamenetén. 
Fejleszteni kell majd olyan szolgáltatásokat, amelyek könnyebbé teszik a nyelv sajátosságainak megértését, például a hivatkozások élettartamának nyomon követését 
(ezeket a sajátosságokat \myaref{sec:rust} fejezet mutatja majd be bővebben), és amelyek figyelmeztetnek a gyakori Rust programozási hibákra.

% TODO: Itt egy olyan leírásnak kellene majd szerepelnie, ami felhívja az olvasó figyelmét, hogy
% - milyen fontos, hogy legyen ilyen IDE,
% - jelenleg még nincs ilyen,
% - a feladat elvégzése nem triviális.

% TODO: Címszavasan bemutatásra kell, hogy kerüljenek a felhasznált technológiák.

% 1-2 oldal elég