\Chapter{Bevezetés}

% TODO: A bevezetésnek egy 1-2 oldalas áttekintésnek kellene lennie (alpontok nélkül), amelyben inkább a motiváció és a témakör aktualitása szerepel csak.

A Rust egy új, hardverközeli nyelv, ami az elmúlt években egyre nagyobb
népszerűségre tett szert.
A nyelvvel ismerkedő programozók között gyakori a más nyelvekben írt
programok újraírása Rust-ban, illetve az új programok létrehozása,
amik kihasználják javukra a Rust egyedi tulajdonságait.
Nyilvánvaló így, hogy szükség van olyan programokra, amik elő- illetve megsegítik
a Rust-ban történő programozást.

A dolgozat célja, hogy a Rust nyelv említett egyedi tulajdonságait
bemutassa az olvasónak, és ezt egy Rust-hoz íródott integrált fejlesztőkörnyezet
létrehozásán keresztül tegye.

Fontos kiemelni, hogy hasonló projektek már léteztek ez előtt:
ilyen például a \textit{SolidOak}\cite{solidoak} illetve a \textit{Ride}\cite{ride}.
Az előző Github oldala archiválva lett, míg az utóbbi fejlesztője
bejelentette, hogy a fejlesztés abbamaradt, így jelenleg a fejlesztőkörnyezet-fejlesztés
Rust-ban egy kiaknázatlan terület.
Továbbá a Rust felhasználói által nyilvántartott \textit{Are we (I)DE yet?}\cite{ideyet} oldal,
ahol összegyűjtik a Rust fejlesztőkörnyezeteket, kódszerkesztőket, és azokat elősegítő programokat,
nem említ más aktív projektet, ami e témában létezne.

A következőekben bemutatom a Rust nyelvet, annak történetét, majd
összehasonlítom a C++ nyelvvel, kiemelve a főbb eltéréseket a kettő között.
A dolgozat feltételezi, hogy az olvasó minimális programozási tudással
rendelkezik, de kiemeli a programozási praktikákat, amik inkább a Rust-ra
jellemzőek, ezáltal tovább folytatva az összehasonlítást az fejlesztőkörnyezet
megvalósítása során.

Ezek után bemutatom egy átlagos fejlesztőkörnyezet összetételét, és
ezek alapján összeállítok egy specifikációt a dolgozathoz mellékelt
program létrehozására.

A megvalósítás során előállítok egy minimális programot, amivel a felhasználó
egy grafikus felület használatával tud kommunikálni,
majd ezt a programot dolgozom tovább egy fejlesztőkörnyezet megközelítéséhez.

Mindezek alatt kiemelem, hogy az éppen használt módszer miért előnyös
vagy hátrányos a Rust nyelvben, milyen alternatíváink léteznek, és
miért nem ezeket az alternatívákat használtam.

Végül, a dolgozat végén tesztként a programot saját magán futtatom, azaz felhasználom
a dolgozat alatt elkészített fejlesztői környezetet, hogy azt
szerkesszem, lefordítsam, és futtassam.

% TODO: Itt egy olyan leírásnak kellene majd szerepelnie, ami felhívja az olvasó figyelmét, hogy
% - milyen fontos, hogy legyen ilyen IDE,
% - jelenleg még nincs ilyen,
% - a feladat elvégzése nem triviális.

% TODO: Címszavasan bemutatásra kell, hogy kerüljenek a felhasznált technológiák.

% 1-2 oldal elég