\chapter*{Felhasználói dokumentáció}

A program képes Rust forráskódok létrehozására, szerkesztésére, illetve Cargo ládák
manipulálására.
A forráskód-szerkesztés és a Cargo parancsok kiadása ládára párhuzamosan léteznek egymás mellett.

Új forrásfájl létrehozásához a felhasználó írhat a program ablakának kódszerkesztő részébe,
majd egy \texttt{Ctrl+S} gomblenyomással, vagy a \textit{Fájl $\,\to\,$ Mentés (Másként)} menüelemmel
elmentheti azt a háttértáron.

Meglévő forrásfájl szerkesztéséhez a \texttt{Ctrl+O} gomblenyomással, vagy a \textit{Fájl $\,\to\,$ Megnyitás}
menüelemmel a felhasználó kiválaszthat egy tetszőleges \texttt{.rs} kiterjesztésű fájlt a háttértáron.
A kiválasztott fájl tartalma bekerül a kódszerkesztő pufferjébe, és szerkeszthetővé válik.

Ha új forrásfájlt szeretnénk létrehozni, de már megnyitottunk egyet, akkor a \texttt{Ctrl+W} gomblenyomással, 
vagy a \textit{Fájl $\,\to\,$ Bezárás} menüelemmel bezárhatjuk a jelenlegi fájlt.
Ha vannak elmentetlen módosításaink, akkor a program rákérdez, hogy szeretnénk-e menteni azokat,
vagy elvetjük őket.
Ugyanez elérhető a \texttt{Ctrl+N} billentyű-kombinációval, illetve a \textit{Fájl $\,\to\,$ Új} menüelemmel.

Új Cargo ládákat a \textit{Cargo $\,\to\,$ Új könyvtár} illetve a \textit{Cargo $\,\to\,$ Új bináris}
menüelemekkel lehet létrehozni, igényeinktől függően.
Létrehozás után a új Cargo láda megnyílik a programban -- természetesen a megnyitott fájllal nem történik semmi.
Ha meglévő ládát szeretnénk megnyitni, akkor válasszuk a \textit{Cargo $\,\to\,$ Megnyitás} lehetőséget.
Egy könyvtárat csak akkor nyithatunk meg ládaként, ha az tartalmaz egy \texttt{Cargo.toml} fájlt!
A megnyitott ládát a \textit{Cargo $\,\to\,$ Bezárás} menüelemmel lehet bezárni.

Láda-megnyitás után, ha a ládánkat le szeretnénk fordítani, akkor válasszuk a \textit{Cargo $\,\to\,$ Fordítás} lehetőséget.
A kódhelyességet a \textit{Cargo $\,\to\,$ Ellenőrzés} menüelemmel tudjuk leellenőriztetni.
Ha a ládánkban léteznek tesztek, akkor azokat a \textit{Cargo $\,\to\,$ Tesztek futtatása} menüelemmel
lehet futtatni.
A \textit{Cargo $\,\to\,$ Takarítás} menüelem kitörli az eddigi fordítások eredményeit, ha arra van szükségünk.
Végül ha bináris ládával dolgozunk, akkor a \textit{Cargo $\,\to\,$ Futtatás} elemmel elindíthatjuk a programunkat.

A programot a \texttt{Ctrl+Q}, vagy a \textit{Fájl $\,\to\,$ Kilépés} lehetőséggel zárhatjuk be.