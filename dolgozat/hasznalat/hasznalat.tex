%Az összefoglaló fejezet
\chapter*{Adathordozó használati útmutató}
\addcontentsline{toc}{chapter}{Adathordozó használati útmutató}

%Ebben a fejezetben kell megadnunk, hogy a szakdolgozathoz mellékelt adathordozót (pl. CD) hogyan lehet elérni, milyen strukturát követ. Minimum 1 maximum 4 oldal a terjedelem. Lehet benne több alszakasz is. A fejezet címe nem módosítható, hasonlóan a következõ részhez (Irodalomjegyzék).

% Dependenciák:
% - Rust: https://www.rust-lang.org/learn/get-started
% - libgtk3-dev, libgtksourceview3-dev, libpango1.0-dev

\noindent A szakdolgozatomhoz mellékelt adathordozó eszközön található adatok struktúrája:

\begin{itemize}
    \item \texttt{dolgozat/}
    \begin{itemize}
        \item \texttt{forras/}: A szakdolgozat \LaTeX{} formátumú forrása, fejezetekre bontva.
        \item \texttt{dolgozat.pdf}: A szakdolgozat megformázott változata.
        \item \texttt{kiiras.rtf}: A szakdolgozathoz járó feladatkiírás.
        \item \texttt{osszefoglalo.tex}: A szakdolgozatból kiemelt \LaTeX{} formátumú összefoglaló magyar része.
        \item \texttt{osszefoglalo.pdf}: A szakdolgozatból kiemelt összefoglaló magyar része, megformázott változatban.
        \item \texttt{summary.tex}: A szakdolgozatból kiemelt \LaTeX{} formátumú összefoglaló angol része.
        \item \texttt{summary.pdf}: A szakdolgozatból kiemelt összefoglaló angol része, megformázott változatban.
    \end{itemize}
    \item \texttt{rozsda\_ide/}: A program, ami a szakdolgozat témájaként szerepelt.
    \begin{itemize}
        \item \texttt{rozsda\_ide}: A program futtatható változata (Linuxon).
        \item \texttt{src/}: A program forráskódja.
    \end{itemize}
\end{itemize}