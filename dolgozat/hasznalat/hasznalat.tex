%Az összefoglaló fejezet
\chapter*{Adathordozó használati útmutató}
\addcontentsline{toc}{chapter}{Adathordozó használati útmutató}

%Ebben a fejezetben kell megadnunk, hogy a szakdolgozathoz mellékelt adathordozót (pl. CD) hogyan lehet elérni, milyen strukturát követ. Minimum 1 maximum 4 oldal a terjedelem. Lehet benne több alszakasz is. A fejezet címe nem módosítható, hasonlóan a következõ részhez (Irodalomjegyzék).

% Dependenciák:
% - Rust: https://www.rust-lang.org/learn/get-started
% - libgtk3-dev, libgtksourceview3-dev, libpango1.0-dev

\noindent A szakdolgozatomhoz mellékelt adathordozó eszközön található adatok struktúrája:

\begin{itemize}
    \item \texttt{dolgozat.pdf}: A teljes szakdolgozat, \LaTeX{} segítségével készítve.
    \item \texttt{felhasznaloi-dokumentacio.pdf}: Felhasználói dokumentáció a program használatához, \LaTeX{}-ben írva.
    \item \texttt{rozsda-ide/}: A program ládája, benne a program forráskódjával.
    \item \texttt{rozsda-ide}: A program futtatható változata, 64-bites Linuxra.
\end{itemize}

\noindent A program futtatásához a következő könyvtárak szükségesek:

\begin{itemize}
    \item \texttt{libgtk-3}, legalább 3.22-es verzió.
    \item \texttt{libgdk-3}
    \item \texttt{libpango-1.0}
    \item \texttt{libcairo-1.0} és \texttt{libpangocairo-1.0}
    \item \texttt{libgtksourceview-3}, legalább 3.18-as verzió.
\end{itemize}

\noindent A forráskód lefordításához a fenti könyvtárak \texttt{-dev} verziója szükséges,
továbbá a következő Rust követelmények:

\begin{itemize}
    \item Rust
    \item \texttt{rustup}
    \item Cargo
\end{itemize}

\noindent A következő paranccsal, illetve a Rust oldalán (\url{rust-lang.org})
beszerezhető mind a három egyszerre: \fbox{\texttt{curl https://sh.rustup.rs -sSf | sh}}